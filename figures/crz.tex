% Decomposition of controlled-R_z gate into CNOT and single-qubit rotations.

\documentclass[twoside]{article}
%\usepackage[dvips]{graphicx}
%\usepackage{times}
%\usepackage{fullpage}
%\usepackage{rotating}
%\usepackage{eepic}
%\usepackage{amsfonts}
%\usepackage{algorithmic}
%\usepackage{amsthm}

%\theoremstyle{plain}
%\newtheorem{theorem}{Theorem}

\newcommand{\targfix}{\qw {\xy {<0em,0em> \ar @{ - } +<.39em,0em>
\ar @{ - } -<.39em,0em> \ar @{ - } +<0em,.39em> \ar @{ - }
-<0em,.39em>},<0em,0em>*{\rule{.01em}{.01em}}*+<.8em>\frm{o}
\endxy}}

\input{Qcircuit}

\newcommand{\braket}[2]{\langle #1|#2 \rangle}
\newcommand{\normtwo}{\frac{1}{\sqrt{2}}}
\newcommand{\norm}[1]{\parallel #1 \parallel}

\newcommand\coolleftbrace[2]{%
#1\left\{\vphantom{\begin{matrix} #2 \end{matrix}} \right.}

\newcommand\coolrightbrace[2]{%
\left. \vphantom{\begin{matrix} #2 \end{matrix}} \right\} #1}

\begin{document}

\begin{displaymath}
\begin{array}{ccc}
\Qcircuit @C=1.5em @R=1.5em {
   & \qw      & \ctrl{1}                   & \qw \\
   & \qw      & \gate{\frac{\pi}{2^{d}}} & \qw \\
 }
&
\begin{array}{c}
\\
\\
\\
= \\
\end{array}
&
\Qcircuit @C=1.5em @R=1.5em {
& \qw & \qw & \qw & \ctrl{1} & \qw & \gate{\frac{\pi}{2^{d+1}}} & \qw & \ctrl{1} & \qw\\
 & \qw & \gate{\frac{\pi}{2^{d+1}}} & \qw & \targfix & \qw & \gate{\frac{\pi}{2^{d+1}}} & \qw & \targfix & \qw
}
\end{array}
%}
\end{displaymath}

\begin{displaymath}
\begin{array}{ccc}
\Qcircuit @C=1.5em @R=1.5em {
   & \ctrl{1}                 & \qw \\
   & \gate{\frac{\pi}{2^{d}}} & \qw \\
   & \qw                      & \qw \\
 }
&
\begin{array}{c}
\\
\\
\\
= \\
\end{array}
&
\Qcircuit @C=0.75em @R=1.5em {
& \qw & \ctrl{1} & \qw & \qw                        & \qw & \ctrl{1} & \qw\\
& \qw & \qswap   & \qw & \qw                        & \qw & \qswap   & \qw\\
& \qw & \qswap \qwx     & \qw & \gate{\frac{\pi}{2^{d}}} & \qw & \qswap \qwx     & \qw
}
\end{array}
\end{displaymath}

\end{document}