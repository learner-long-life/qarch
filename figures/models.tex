% a sample file for Journal of Quantum Information and Computation (QIC) in 
% LaTex2e by inputing macro file "qic.sty" with command \usepackage{qic}, 
% all the macros have been defined in the style file, so it is no need to 
% put many macros at the beginning of the text file  

\documentclass[twoside]{article}
%\usepackage[dvips]{graphicx}
%\usepackage{times}
%\usepackage{fullpage}
%\usepackage{rotating}
%\usepackage{eepic}
%\usepackage{amsfonts}
%\usepackage{algorithmic}
%\usepackage{amsthm}

%\theoremstyle{plain}
%\newtheorem{theorem}{Theorem}

\newcommand{\targfix}{\qw {\xy {<0em,0em> \ar @{ - } +<.39em,0em>
\ar @{ - } -<.39em,0em> \ar @{ - } +<0em,.39em> \ar @{ - }
-<0em,.39em>},<0em,0em>*{\rule{.01em}{.01em}}*+<.8em>\frm{o}
\endxy}}

\input{Qcircuit}

\newcommand{\braket}[2]{\langle #1|#2 \rangle}
\newcommand{\normtwo}{\frac{1}{\sqrt{2}}}
\newcommand{\norm}[1]{\parallel #1 \parallel}

\newcommand\coolleftbrace[2]{%
#1\left\{\vphantom{\begin{matrix} #2 \end{matrix}}\right.}

\begin{document}

% AC
\begin{displaymath}
\Qcircuit @C=1em @R=1em {
\lstick{\ket{a}}	& \qw      & \ctrl{8} & \qw \\
\lstick{\ket{b}}    & \gate{H} & \qw      & \qw \\
\lstick{\ket{c}}    & \qw      & \qw      & \qw \\
\lstick{\ket{d}}    & \gate{H} & \qw      & \qw \\
\lstick{\ket{e}}    & \qw      & \qw      & \qw \\
\lstick{\ket{f}}    & \gate{H} & \qw      & \qw \\
\lstick{\ket{g}}    & \qw      & \qw      & \qw \\
\lstick{\ket{h}}    & \gate{H} & \qw      & \qw \\
\lstick{\ket{i}}    & \qw      & \targfix & \qw \\
}
\end{displaymath}

% 1D NTC
\begin{displaymath}
\Qcircuit @C=1em @R=1em {
\lstick{\ket{a}}	& \qw      & \qw & \ctrl{1} & \qw      & \qw      & \qw      & \qw      & \qw      & \qw      & \qw      & \qw \\
\lstick{\ket{b}}    & \gate{H} & \qw & \targfix & \ctrl{1} & \qw      & \qw      & \qw      & \qw      & \qw      & \qw      & \qw \\
\lstick{\ket{c}}    & \qw      & \qw & \qw      & \targfix & \ctrl{1} & \qw      & \qw      & \qw      & \qw      & \qw      & \qw \\
\lstick{\ket{d}}    & \gate{H} & \qw & \qw      & \qw      & \targfix & \ctrl{1} & \qw      & \qw      & \qw      & \qw      & \qw \\
\lstick{\ket{e}}    & \qw      & \qw & \qw      & \qw      & \qw      & \targfix & \ctrl{1} & \qw      & \qw      & \qw      & \qw \\
\lstick{\ket{f}}    & \gate{H} & \qw & \qw      & \qw      & \qw      & \qw      & \targfix & \ctrl{1} & \qw      & \qw      & \qw \\
\lstick{\ket{g}}    & \qw      & \qw & \qw      & \qw      & \qw      & \qw      & \qw      & \targfix & \ctrl{1} & \qw      & \qw \\
\lstick{\ket{h}}    & \gate{H} & \qw & \qw      & \qw      & \qw      & \qw      & \qw      & \qw      & \targfix & \ctrl{1} & \qw \\
\lstick{\ket{i}}    & \qw      & \qw & \qw      & \qw      & \qw      & \qw      & \qw      & \qw      & \qw      & \targfix & \qw \\
}
\end{displaymath}

% 2D NTC
\begin{displaymath}
\Qcircuit @C=1em @R=1em {
\lstick{\ket{a}}	& \qw      & \qw & \ctrl{1} & \qw      & \qw      & \qw      & \qw & \qw \\
\lstick{\ket{b}}    & \gate{H} & \qw & \targfix & \ctrl{3} & \qw      & \qw      & \qw & \qw \\
\lstick{\ket{c}}    & \qw      & \qw & \qw      & \qw      & \qw      & \qw      & \qw & \qw \\
\lstick{\ket{d}}    & \gate{H} & \qw & \qw      & \qw      & \qw      & \qw      & \qw & \qw \\
\lstick{\ket{e}}    & \qw      & \qw & \qw      & \targfix & \ctrl{3} & \qw      & \qw & \qw \\
\lstick{\ket{f}}    & \gate{H} & \qw & \qw      & \qw      & \qw      & \qw      & \qw & \qw \\
\lstick{\ket{g}}    & \qw      & \qw & \qw      & \qw      & \qw      & \qw      & \qw & \qw \\
\lstick{\ket{h}}    & \gate{H} & \qw & \qw      & \qw      & \targfix & \ctrl{1} & \qw & \qw \\
\lstick{\ket{i}}    & \qw      & \qw & \qw      & \qw      & \qw      & \targfix & \qw & \qw \\
}
\end{displaymath}

\end{document}